\documentclass[a4paper,12pt]{report}
\usepackage[T1,T2A]{fontenc}
\usepackage[utf8]{inputenc}
\usepackage[russian]{babel}
\usepackage{wrapfig}
\usepackage{hyperref}
\usepackage{graphicx}
\usepackage{listings}

\lstset{ %
flexiblecolumns=false,
language=C++,                % choose the language of the code
basicstyle=\footnotesize,       % the size of the fonts that are used for the code
showspaces=false,               % show spaces adding particular underscores
showstringspaces=false,         % underline spaces within strings
showtabs=false,                 % show tabs within strings adding particular underscores
frame=none,           % adds a frame around the code
tabsize=2,          % sets default tabsize to 2 spaces
captionpos=b,           % sets the caption-position to bottom
breaklines=false,        % sets automatic line breaking
breakatwhitespace=false,    % sets if automatic breaks should only happen at whitespace
escapeinside={\%*}{*)}          % if you want to add a comment within your code
}

\graphicspath{{img/}}

\newcommand{\fotocake}{ФОТОНАТОРТЕ.РФ}

\title{{\Huge Инструкция по установке системы}}
\author{\fotocake}

\begin{document}

	\maketitle
	\tableofcontents

	\chapter*{Введение}
	\addcontentsline{toc}{chapter}{Введение}
	
		Данный документ представляет собой руководство по установке трех основных частетй системы \fotocake:
		
		\begin{itemize}
			\item Приложение с конструктором тортов.			
			\item Панель управления заказами и настройками конструктора.
			\item Сервер предоставляющий REST интерфейс управления данными системы.
		\end{itemize}
		
		Для установки Приложения и Панели управления вам потребуется веб-сервер. Например Apache, nginx или любой другой, поддерживающий выдачу статического контента.
		
		Для работы REST-сервера кроме веб-сервера необходим инетпретатор PHP поддерживающий интерфейс Fast-CGI, а также сервер базы данных MongoDB.
		
		Инструкции по установке веб-сервера, PHP и MongoDB можно найти на официальных сайтах.
		
		В качестве веб-сервера данном руководстве будет использоваться nginx.
		
		Так же для установки вам может понадобиться система контроля версий GIT.
		
	\chapter*{Приложение и Панель управления}
	\addcontentsline{toc}{chapter}{Приложение и Панель управления}
	
		\section*{Приложение}
		\addcontentsline{toc}{section}{Приложение}
		
			Для установки приложения из исходных кодов необходимо склонировать проект приложения из репозитория в необходимую папку (например \tt{/var/www/fotonatorte.ru}).
		

			\tt{\begin{lstlisting}[caption={Клонирование проекта}]
git clone git@github.com:kononencheg/Photo-Cake.git /var/www/fotonatorte.ru
			\end{lstlisting}}
			
	
		Для использования  приложения нужно сконфигурировать веб-сервер для выдачи контента по выбранному адресу из папки \tt{public} проекта приложения.
	
		Добавим в файл \tt{nginx.conf} настроек сервера в блок \tt{http} следующие строки:
	
		\tt{\begin{lstlisting}[caption={Конфигурация веб-сервера приложения}]
	server {
		listen				fotonatorte.ru:80;
		server_name		fotonatorte.ru;
		
		charset		utf-8;
		
		root	/var/www/fotonatorte.ru/public;
	}
		\end{lstlisting}}
	
		Если ваш сервер отвечает на запросы к \tt{fotonatorte.ru}  приложение будет доступно по адресу \tt{http://fotonatorte.ru/}.
		
		\section*{Панель управления}
		\addcontentsline{toc}{section}{Панель управления}
	
			Установка Панели управления принципиально не отличается от установки Приложением и не будет описываться отдельно.

	\chapter*{REST-сервер}
	\addcontentsline{toc}{chapter}{REST-сервер}
	
		Для установки REST-сервера неоходимо, так же сначала склонировать проект по примеру из листинга 1. 
		
		Пример конфигурации веб-сервера для REST интерфейса (файл \tt{nginx.conf}):
		
		\tt{\begin{lstlisting}[caption={Конфигурация REST-сервера}]
server {
	listen          api.fotonatorte.ru:80;
	server_name     api.fotonatorte.ru;

	charset     utf-8;

	root    /var/www/api.fotonatorte.ru/public;
	index   index.php;

	location ~.php$ {
		fastcgi_pass 127.0.0.1:9000;
		
		fastcgi_index index.php;
		
		fastcgi_param SCRIPT_FILENAME /var/www/api.fotonatorte.ru/public/
$fastcgi_script_name;

		fastcgi_param PATH_INFO $fastcgi_script_name;  
		fastcgi_param APPLICATION_ENV production;

		include fastcgi_params;
	}
}
		\end{lstlisting}}
		
		Параметр перенаправления к Fast-CGI \verb APPLICATION_ENV  устанавливает значение переменной окружения в зависимости от использования приложения может иметь следующие значения:
		
		\begin{description}
			\item[development] -- разработка.		
			\item[staging] -- тестирование.
			\item[production] -- эксплуатация.
		\end{description}
		
		Если все было установлено без ошибок то в ответ на пустой запрос к API будет выдан следующий ответ: 
		
		\tt{\begin{lstlisting}[caption={Ответ на запрос \tt{http://api.fotonatorte.ru/}}]
{"errors":[{"message":"Unknown method calling","code":404}]}
		\end{lstlisting}}
		
		\section*{Файлы конфигурации}
		\addcontentsline{toc}{section}{Файлы конфигурации}	
		
		Параметры конфигурации REST-сервера и панели управления находятся в папке \tt{config} проекта сервера в файлах \tt{application.ini} и \tt{admin-panel.ini}.
		
			\subsection*{Настройки REST-сервера}
			\addcontentsline{toc}{subsection}{Настройки REST-сервера}	

		
			\tt{\begin{lstlisting}[caption={Настройки REST-сервера}]
	[production]

	mongo.db = "cakes" # Database name

	files.folder = "/var/db/www/files/" 				# Files folder 
	files.url = "http://files.fotonatorte.ru/"	# Files folder url

	[staging:production]

	mongo.db = "test_cakes"

	[development:production]

	files.url = "http://files.fotonatorte.local/"
			\end{lstlisting}}		
		
			Настройки устанавливают параметры REST сервера в зависимости от текущей переменной окружения. На данный момент файл настроек устанавливает следующие параметры: 
		
			\begin{description}
				\item[mongo.db] -- Имя базы для работы.		
				\item[files.folder] -- Папка для загрузки файлов.
				\item[files.url] -- URL папки с файлами.
			\end{description}
		
			\subsection*{Настройки панели управления}
			\addcontentsline{toc}{subsection}{Настройки панели управления}	

		
			\tt{\begin{lstlisting}[caption={Настройки панели управления}]
[production]

app["site"]   = "http://fotonatorte.ru/"
app["vk"]     = "http://fotonatorte.ru/vk.html"
app["ok"]     = "http://fotonatorte.ru/ok.html"
app["bakery"] = "http://fotonatorte.ru/bakery.html"

role[0] = "Admin"
role[1] = "Bakery"

order_status[0] = "New"
order_status[1] = "Approved"
order_status[2] = "Declined"

delivery_status[0] = "New"
delivery_status[1] = "In progress"
delivery_status[2] = "Ready"

payment_status[0] = "New"
payment_status[1] = "Paid"

shape["round"] = "Round"
shape["rect"] = "Rect"

[staging:production]

app["site"]   = "http://qa.fotonatorte.ru/"
app["vk"]     = "http://qa.fotonatorte.ru/vk.html"
app["ok"]     = "http://qa.fotonatorte.ru/ok.html"
app["bakery"] = "http://qa.fotonatorte.ru/bakery.html"

[development:production]

app["site"]   = "http://fotonatorte.local/"
app["vk"]     = "http://fotonatorte.local/vk.html"
app["ok"]     = "http://fotonatorte.local/ok.html"
app["bakery"] = "http://fotonatorte.local/bakery.html"
			\end{lstlisting}}		
		
			Настройки панели управления содержат описание ссылок на различные типы приложения:
			
			\begin{itemize}
				\item Сайт;	
				\item Приложение во Вконтакте;				
				\item Приложение в Одноклассах;		
				\item Приложение для конкретной кондитерсокой.
			\end{itemize}
			
			А также содержат таблицы сообвествия кодов сущностей с их именами в приложении.
			
\end{document}
	
	
